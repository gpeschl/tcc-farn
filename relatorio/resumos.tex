%
% ********** Resumo
%

% Usa-se \chapter*, e n�o \chapter, porque este "cap�tulo" n�o deve
% ser numerado.
% Na maioria das vezes, ao inv�s dos comandos LaTeX \chapter e \chapter*,
% deve-se usar as nossas vers�es definidas no arquivo comandos.tex,
% \mychapter e \mychapterast. Isto porque os comandos LaTeX t�m um erro
% que faz com que eles sempre coloquem o n�mero da p�gina no rodap� na
% primeira p�gina do cap�tulo, mesmo que o estilo que estejamos usando
% para numera��o seja outro.
\mychapterast{Resumo}

Neste trabalho ser� relatado o desenvolvimento de um cliente de \textit{Web Service} baseado em SOAP (\textit{Simple Object Access Protocol}) que deve realizar v�rias requisi��es para sincronizar uma grande massa de dados em um tempo relativamente curto. Foram abordadas tr�s t�cnicas: requisi��es sob demanda do usu�rio; atualizador com requisi��es s�ncronas e sequenciais em um �nico processo; e atualizador com requisi��es s�ncronas em v�rios processos.

\vspace{1.5ex}

{\bf Palavras-chave}: \textit{Web services}, SOAP, computa��o distribu�da, programa��o concorrente.
%
% ********** Abstract
%

\mychapterast{Abstract}

On the following work, will be reported the development of a Web Service client based on SOAP that should make many requests to synchronize a huge data in a short time. Three techniques were considered: request on user demand; updater with synchronous and sequential requests in a single process; and updater with synchronous requests with multiprocessing.

\vspace{1.5ex}

{\bf Keywords}: Web Services, SOAP, distributed computing, multiprocessing.
