%%
%% Captulo 1: Modelo de Captulo
%%

% Est sendo usando o comando \mychapter, que foi definido no arquivo
% comandos.tex. Este comando \mychapter  essencialmente o mesmo que o
% comando \chapter, com a diferena que acrescenta um \thispagestyle{empty}
% aps o \chapter. Isto é necessário para corrigir um erro de LaTeX, que
% coloca um número de pgina no rodapé de todas as páginas iniciais dos
% capítulos, mesmo quando o estilo de numeração escolhido é outro.
\mychapter{Introdução}
\label{Cap:introducao}

Este trabalho trata de um cliente de \textit{Web Service} SOAP que deve obter o mais rapidamente possível uma grande quantidade de dados do servidor e persistí-los em seu banco de dados. Existem inúmeras formas de resolver esse problema. Neste trabalho serão abordadas algumas técnicas possíveis.

\section{Descrição geral}

O trabalho é baseado em um \textit{Web Service} SOAP, que é um protocolo para troca de informações estruturadas em uma plataforma descentralizada e distribuída.

Um servidor é cliente dos dados de outro e precisa fornecer estes dados o mais rapidamente possível. Para não fornecer informação atrasada, ele precisa requisitar ao fornecedor de dados o mais frequentemente possível ao mesmo tempo que tem que manter um bom tempo de resposta para seus usuários.

No problema serão abordadas técnicas de otimização de I/O e comunicação através de HTTP e SOAP.

\section{Objetivo}

O objetivo do trabalho é resolver os problemas relacionados à otimização do tempo de resposta e à frequência de atualização do dado. Por se tratar de um problema de otimização de tempo de resposta é necessário pensar no tempo de resposta de uma requisição HTTP e no overhead causado por ela e analisar a arquitetura de um processador para otimizar o processamento de tais requisições.

\section{Motivação}

Com o avanço da informática e de conceitos como a computação em nuvem, cada vez mais os servidores estão em constante comunicação ao redor do mundo. Seja para autenticar usuários, seja para persistir dados ou seja para solicitar dados. Um portal de notícias, por exemplo, exibe, além das notícias, dados meteorológicos e econômicos. Esses dados normalmente não são produzidos pelo próprio portal e então são necessárias parcerias com fornecedores de conteúdo que disponham de tal informação. Estes fornecedores, em geral, disponibilizam os dados através de um \textit{Web Service}.

Dado que no mundo jornalístico a agilidade na entrega de informações é um fator primordial para o sucesso, surge a necessidade de otimizar o tempo de consumo de tais fornecedores a fim de entregar aos usuários a informação mais recente possível.

\section{Metodologia}

O projeto foi dividido em três partes: 

\begin{itemize}
\item módulo para realizar comunicação utilizando SOAP;
\item estratégia de atualização de dados sob demanda;
\item atualizador de dados independente de este ser requisitado ou não.
\end{itemize}


\subsection{Módulo para realizar comunicação utilizando SOAP}

Foi necessário desenvolver um módulo capaz de fazer a comunicação utilizando SOAP, serví-los aos clientes e persistí-los em seu banco de dados.

\subsection{Estratégia de atualização de dados sob demanda}

Uma das possíveis opções de atualização de dados é fazê-lo sob demanda. A medida em que um usuário solicita o dado, o portal requisita-o ao fornecedor e gera a resposta para o usuário. Esta foi a primeira estratégia analisada.

\subsection{Atualizador de dados independente de este ser requisitado ou não}

A atualização de dados sob demanda leva a um tempo de resposta muito alto para o usuário. Então foi analisada uma nova estratégia de atualizar os dados indepentementemente de eles terem sido requisitados pelos usuários ou não. Esta estratégia se mostrou mais interessante por reduzir o tempo de resposta para o usuário mas expôs bons desafios.

