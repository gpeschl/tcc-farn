\mychapter{\textit{Web Services}}
\label{Cap:web_services}

\textit{Web service} é uma interface acessível através da rede para uma funcionalidade de uma aplicação construída usando tecnologias definidas como padrão na Internet \cite{ref-oreilly-soap}. É apenas mais uma camada de troca de mensagens entre uma aplicação e outra. A principal vantagem no seu uso é prover comunicação entre diferentes aplicações independente da plataforma ou linguagem de programação utilizada por elas, o que garante interoperabilidade entre sistemas desde que ambos utilizem o mesmo protocolo de comunicação entre si.

Através do uso de \textit{web services} é possível fazer com que uma aplicação faça chamadas de métodos remotos de outra aplicação. É simples como gerar uma requisição que encapsule qual método será chamado e quais os parâmetros e esperar uma resposta com o resultado da chamada do método. Para isso se faz necessário quebrar a camada de aplicação da pilha de camadas do modelo de redes TCP/IP em quatro camadas: aplicação, descoberta, descrição e empacotamento.

\section{Aplicação}

A camada de aplicação é o código que precisará se comunicar com a outra aplicação através do \textit{web service}, que pode ser escrito em qualquer linguagem.

\section{Descoberta}

Esta camada é responsável por disponibilizar metadados sobre \textit{web services} de modo a facilitar a busca pela aplicação necessária para cada projeto.

\section{Descrição}

A descrição define como o \textid{web service} deve ser utilizado. Quais os métodos que ele expõe, quais os parâmetros necessários em cada um deles e quais as possíveis respostas. O \textit{Web Service Description Language} (WSDL) é um padrão muito utilizado para descrever um \textit{web service} através do uso de XML.

\section{Empacotamento}



